\section{Introduction}
Language modeling task consists in predicting the probability of the next word.
It is a crucial component in real-world applications such as Machine Translation 
and Automatic Speech Recognition. For example, a translation system generates multiple 
translations for the same sentence and the language model scores all the sentences 
and decides the most likely one.

More formally, Language Models assign a probability to a sequence of words: 
given a text corpus with a vocabulary $V$ and a sequence of words $w_1,w_2, \dots, w_{t-1}$, 
we need to compute the probability distribution of the next word $w_t$ \cite{LM_definition}:
\begin{equation}
    P(w_{t} | w_1, w_2, \dots, w_{t-1})
\end{equation}
where $w_t$ can be any word in the vocabulary $V$.
Language models can operate at different levels: character level, n-gram level, 
sentence level or paragraph level.

A recurrent neural network (RNN) is a type of artificial neural network that 
operates on sequences of data of variable length. In the case of RNN, the output 
from the previous step is fed as input to the current step. This is particularly 
useful in applications where there is the need to remember the previous state.

Recurrent Neural Net Language Model (RNNLM) is a type of neural network language 
model that exploits a RNN cell. There are several different types of RNN cells, % TODO: reference
but we will focus on Long-Short Term Memory \cite{LSTM} and Gated Recurrent Unit 
\cite{GRU} cells. Since a RNN can deal with the variable length data, it is 
suitable for modeling data such as sentences in natural language.

